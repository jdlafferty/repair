% !TEX root = ./repair.tex


\def\L{{\mathcal L}}
\section{Repair of neural networks}
\label{sec:neural}

In this section we show how to use robust regression to repair neural networks.
We consider a neural network with one hidden layer,
$$f(x)=\frac{1}{\sqrt{p}}\sum_{j=1}^p\beta_j\psi(W_j^Tx),$$
where $\psi$ is either the rectified linear unit (ReLU) function $\psi(t)=\max(t,0)$, or the
hyperbolic tangent $\psi(t) = \tanh(t) = \frac{e^t - e^{-t}}{e^t + e^{-t}}$.
The factor $p^{-1/2}$ in the definition above is convenient for our theoretical analysis.
In this section we present the analysis for the hyperbolic tangent activation function, with the ReLU
deferred to the appendix.

With the squared error loss function
$$\L(\beta,W)=\frac{1}{2}\sum_{i=1}^n\left(y_i - \frac{1}{\sqrt{p}}\sum_{j=1}^p\beta_j\psi(W_j^Tx_i)\right)^2,$$
we consider training the model using a standard gradient descent algorithm (Algorithm \ref{alg:GD}).

\vskip10pt
\begin{algorithm}[H]
\DontPrintSemicolon
%\SetKwInOut{Input}{Input}\SetKwInOut{Output}{Output}
\vskip5pt
\nl Input: Data $(y, X)$ and the number of iterations $t_{\max}$. \\[3pt]
\nl Initialization: $W_j(0)\sim N(0,d^{-1}I_d)$ and $\beta_j(0)\sim N(0,1)$ independently for all $j\in[p]$. \\[3pt]
\nl Iterate: For $t$ in $1:t_{\max}$, compute
\begin{align*}
\beta_j(t) &= \beta_j(t-1) - \left.\gamma\frac{\partial \L(\beta,W)}{\partial \beta_j}\right|_{(\beta,W)=(\beta(t-1),W(t-1))} \quad j\in[p], \\
W_j(t) &= W_j(t-1) - \left.\frac{\gamma}{d}\frac{\partial \L(\beta,W)}{\partial W_j}\right|_{(\beta,W)=(\beta(t),W(t-1))} \quad j\in[p].
\end{align*}
\nl Output: Trained parameters $\beta(t_{\max})$ and $W(t_{\max})$.\\[5pt]
\caption{Gradient descent for neural nets}\label{alg:GD}
\end{algorithm}
\vskip10pt

Based on this standard gradient descent algorithm, we consider two estimators of the parameters $(\wh{\beta},\wh{W})$.
The first is simply to set $\wh{\beta}=\beta(t_{\max})$ and $\wh{W}=W(t_{\max})$; this is the usual estimator.

In the second approach, one fixes $\wh{W} = W(t_{\max})$ and then retrains $\beta$
using gradient descent for the objective $\|y-\psi(X\wh{W})\beta\|^2$ after initializing at zero.
In this way, $\wh{\beta}$ is an approximation to the minimal $\ell_2$ norm solution of $\|y-\psi(X\wh{W})\beta\|^2$.
This estimator can be viewed as a linear model that uses features extracted from the data by the neural network.

Now consider the contaminated model $\eta=\wh{\beta}+z$ and $\Theta_j=\wh{W}_j+Z_j$, where each entry of $z$ and $Z_j$ is zero with probability $1-\epsilon$ and follows an arbitrary distribution with the complementary probability $\epsilon$. We analyze the following repair algorithm.

\vskip10pt
\begin{algorithm}[H]
\DontPrintSemicolon
\vskip5pt
\nl Input: Contaminated model $(\eta,\Theta)$, design matrix $X$, and initializations $\beta(0)$, $W(0)$. \\[3pt]
\nl Repair of the hidden layer: For $j\in[p]$, compute
$$\wt{v}_j=\argmin_v\|\Theta_j-W_j(0)-X^Tv_j\|_1,$$
and set $\wt{W}_j=W_j(0)+X^T\wt{v}_j$. \\[8pt]
\nl Repair of the output layer: Compute
$$\wt{u}=\argmin_u\|\eta-\beta(0)-\psi(\wt{W}^TX^T)u\|_1,$$
and set $\wt{\beta}=\beta(0)+\psi(\wt{W}^TX^T)\wt{u}$. \\[3pt]
\nl Output: The repaired parameters $\wt{\beta}$ and $\wt{W}$. \\[5pt]
\caption{Model repair for neural networks}\label{alg:MR}
\end{algorithm}
\vskip10pt

\begin{remark}
Algorithm \ref{alg:MR} adopts a layerwise repair strategy. This algorithm extends naturally to multilayer networks, repairing the parameters in stages with a forward pass through the layers. We leave the multilayer extension of our analysis to future work.
\end{remark}

\begin{remark}
It is important to note that the repair of neural networks not only requires $X$, but it also
requires the initializations $\beta(0)$ and $W(0)$. From a practical
perspective, this can be easily achieved by setting a seed using a pseudorandom number generator to initialize the
parameters, and making the seed available to the repair algorithm. We also note that when $\hat\beta$ is
trained after fixing $\hat W$, one can replace $\beta(0)$ by 0 in Algorithm \ref{alg:MR}.
\end{remark}

Since the gradient $\frac{\partial \L(\beta,W)}{\partial W_j}$ lies in the row space of $X$, the vector $\wh{W}_j-W_j(0)$ also lies in the row space of $X$. Thus, the theoretical guarantee of the hidden layer repair directly follows
Corollary \ref{cor:repair-linear}.
The repair of the output layer is more complicated, because the gradient $\frac{\partial \L(\beta,W)}{\partial \beta_j}|_{W=W(t-1)}$ lies in the row space of $\psi(XW(t-1))$, which changes over time. Thus, we cannot directly apply the result of Corollary \ref{cor:repair-rf} for the random feature model. However, when the neural network is overparametrized, it can be shown that the gradient descent algorithm (Algorithm \ref{alg:GD}) leads to $W(t)$ that is close to the initialization $W(0)$ for all $t\geq 0$. We establish this result in the following theorem by assuming that $x_i$ is i.i.d. $N(0,I_d)$ and $|y_i|\leq 1$ for all $i\in[n]$. Define $u(t)\in\mathbb{R}^n$ with its $i$th entry given by
$u_i(t)=\frac{1}{\sqrt{p}}\sum_{j=1}^p\beta_j(t)\psi(W_j(t)^Tx_i)$.
\vskip10pt
\begin{thm}\label{thm:nn-grad}
Assume $\frac{n}{d}$, $\frac{n^3(\log p)^2}{p}$, and $\gamma\left(1+\frac{n^4(\log p)^2}{p}\right)$ are all sufficiently small. Then, we have
\begin{equation}
\|y-u(t)\|^2 \leq \left(1-\frac{\gamma}{8}\right)^t\|y-u(0)\|^2, \label{eq:iter-function}
\end{equation}
and
\begin{eqnarray}
\label{eq:iter-parameter} \max_{1\leq j\leq p}\|W_j(t)-W_j(0)\| &\leq& R_1, \\
\label{eq:iter-parameter-beta} \max_{1\leq j\leq p}|\beta_j(t)-\beta_j(0)| &\leq& R_2,
\end{eqnarray}
for all $t\geq 1$ with high probability, where $R_1=\frac{100n\log p}{\sqrt{pd}}$ and $R_2=32\sqrt{\frac{n^2\log p}{p}}$.
\end{thm}

Theorem \ref{thm:nn-grad} assumes that the width of the neural network is wide compared with the sample size in the sense that $\frac{p}{(\log p)^4}\gg n^3$. For fixed $n$, the limit of the neural network when $p\rightarrow\infty$ is known as the neural tangent kernel (NTK) regime, and the behavior of gradient descent under this limit has been studied by \cite{jacot2018neural}. The result of Theorem \ref{thm:nn-grad} follows the explicit calculation in \cite{du2018gradient}, and we are able to sharpen some of the asymptotic conditions in \cite{du2018gradient}.

The theorem has two conclusions. The first conclusion shows the gradient descent algorithm has global convergence in the sense of (\ref{eq:iter-function}) even though the loss $\L(\beta,W)$ is nonconvex. The second conclusion shows that the trajectory of the algorithm $(W(t),\beta(t))$ is bounded within some radius of the initialization. This allows us to characterize the repaired model $\wt{\beta}$ for the output layer.

Let us first consider the case $\wh{\beta}=\beta(t_{\max})$ and $\wh{W}=W(t_{\max})$. Since the vector $\beta(t)-\beta(t-1)$ lies in the row space of $\psi(XW(t-1))$ for every $t$, one can show that $\wh{\beta}-\beta(0)$ approximately lies in the row space of $\psi(XW(0))$ by Theorem \ref{thm:nn-grad}. Therefore, by extending the result of Corollary \ref{cor:repair-rf} that includes the bias induced by the row space approximation, we are able to obtain the following guarantee for the model repair.





\vskip10pt
\begin{thm}\label{thm:repair-nn-1}
Under the conditions of Theorem \ref{thm:nn-grad}, additionally assume that $\frac{\log p}{d}$, $\frac{\sqrt{\frac{n}{d}\log\left(\frac{ed}{n}\right)}}{1-\epsilon}$ and $\frac{n^2\log p}{p(1-\epsilon)}$ are sufficiently small. We then have $\wt{W}=\wh{W}$ and $\frac{1}{p}\|\wt{\beta}-\wh{\beta}\|^2 \lesssim \frac{n^2\log p}{p(1-\epsilon)}$ with high probability.
\end{thm}


We also consider the case where $\wh{W}=W(t_{\max})$ and $\wh{\beta}$ is obtained by retraining $\beta$ using the features $\psi(X\wh{W})$. Recall that in this case we shall replace $\beta(0)$ by $0$ in Algorithm \ref{alg:MR}. Note that the vector $\wh{\beta}$ exactly lies in the row space of $\psi(X\wh{W})$. This allows us to extend the result of Lemma \ref{lem:design-rf} to the matrix $\psi(\wh{W}^TX^T)$ with the help of Theorem \ref{thm:nn-grad}. Then, we can directly apply Theorem \ref{thm:robust-reg}. We are able to obtain exact recovery of both $\wh{\beta}$ and $\wh{W}$ in this case.

\vskip10pt
\begin{thm}\label{thm:repair-nn-2}
Under the conditions of Theorem \ref{thm:nn-grad}, additionally assume that $\frac{\log p}{d}$, $\frac{\sqrt{\frac{n}{d}\log\left(\frac{ed}{n}\right)}}{1-\epsilon}$, $\frac{n\log p}{p(1-\epsilon)}$ and $\frac{n}{p}\left(\frac{\log p}{1-\epsilon}\right)^{4/3}$ are sufficiently small. We then have $\wt{W}=\wh{W}$ and $\wt{\beta}=\wh{\beta}$ with high probability.
\end{thm}
\vskip10pt

\begin{remark}
As long as the rate that $\epsilon$ tends to $1$ is not so fast, the conditions of Theorem \ref{thm:repair-nn-1} and Theorem \ref{thm:repair-nn-2} can be simplified to $p\gg n^3$ and $d\gg n$ by ignoring the logarithmic factors. The condition $p\gg n^3$ ensures the good property of gradient descent in the NTK regime, but our experimental results show that it can potentially be weakened by an improved analysis.
\end{remark}

\iffalse
\begin{remark}
When the nonlinear activation is replaced by ReLU $\psi(t)=\max(t,0)$, \nb{add text}
\end{remark}
\fi


\section{Proofs of Theorem \ref{thm:repair-nn-1} and Theorem \ref{thm:repair-nn-2}}

To prove Theorem \ref{thm:repair-nn-1}, we need to extend Theorem \ref{thm:main-improved}. Consider $\eta=b+Au^*+z\in\mathbb{R}^m$, where the noise vector $z$ satisfies (\ref{eq:noise-add-con}), and $b\in\mathbb{R}^m$ is an arbitrary bias vector. Then, the estimator $\wh{u}=\argmin_{u\in\mathbb{R}^k}\|\eta-Au\|_1$ satisfies the following theoretical guarantee.
\begin{thm}\label{thm:robust-reg-b}
Assume the design matrix $A$ satisfies Condition A and Condition B. Then, as long as $\frac{\overline{\lambda}\sqrt{\frac{k}{m}\log\left(\frac{em}{k}\right)}+\epsilon\sigma\sqrt{\frac{k}{m}}}{\underline{\lambda}(1-\epsilon)}$ is sufficiently small and $\frac{8\frac{1}{m}\sum_{i=1}^m|b_i|}{\underline{\lambda}(1-\epsilon)}<1$, we have
$$\|\wh{u}-u^*\|\leq \frac{4\frac{1}{m}\sum_{i=1}^m|b_i|}{\underline{\lambda}(1-\epsilon)},$$
with high probability.
\end{thm}
It is easy to see that Theorem \ref{thm:main-improved} is a special case when $b=0$.
Now we are ready to prove Theorem \ref{thm:repair-nn-1}.
\begin{proof}[Proof of Theorem \ref{thm:repair-nn-1}]
We first analyze $\wh{v}_1,...,\wh{v}_p$. The idea is to apply the result of Theorem \ref{thm:main-improved} to each of the $p$ robust regression problems. Thus, it suffices to check if the conditions of Theorem \ref{thm:main-improved} hold for the $p$ regression problems simultaneously. Since the $p$ regression problems share the same Gaussian design matrix, Lemma \ref{lem:design-linear} implies that Conditions A and B hold for all the $p$ regression problems. Next, by scrutinizing the proof of Theorem \ref{thm:main-improved}, the randomness of the conclusion is from the noise vector $Z_j$ through the empirical process bound given by Lemma \ref{lem:EP}. With an additional union bound argument applied to (\ref{eq:double-ub}), Lemma \ref{lem:EP} can be extended to $Z_j$ for all $j\in[p]$ with an additional assumption that $\frac{\log p}{d}$ is sufficiently small. Then, by the same argument in the proof of Corollary \ref{cor:repair-linear}, we have $\wt{W}_j=\wh{W}_j$ for all $j\in[p]$ with high probability.

To analyze $\wh{u}$, we apply Theorem \ref{thm:robust-reg-b}. Note that
\begin{eqnarray*}
\eta_j - \beta_j(0) &=& \beta_j(t_{\max}) - \beta_j(0) +z_j \\
&=& \sum_{t=0}^{t_{\max}-1}\left(\beta_j(t+1)-\beta_j(t)\right) + z_j \\
&=& \frac{\gamma}{\sqrt{p}}\sum_{t=0}^{t_{\max}-1}\sum_{i=1}^n(y_i-u_i(t))\psi(W_j(t)^Tx_i) + z_j \\
&=& \frac{\gamma}{\sqrt{p}}\sum_{t=0}^{t_{\max}-1}\sum_{i=1}^n(y_i-u_i(t))(\psi(W_j(t)^Tx_i)-\psi(W_j(0)^Tx_i)) \\
&& + \frac{\gamma}{\sqrt{p}}\sum_{t=0}^{t_{\max}-1}\sum_{i=1}^n(y_i-u_i(t))\psi(W_j(0)^Tx_i) + z_j.
\end{eqnarray*}
Thus, in the framework of Theorem \ref{thm:robust-reg-b}, we can view $\eta-\beta(0)$ as the response, $\psi(X^TW(0)^T)$ as the design, $z$ as the noise, and $b_j=\frac{\gamma}{\sqrt{p}}\sum_{t=0}^{t_{\max}-1}\sum_{i=1}^n(y_i-u_i(t))(\psi(W_j(t)^Tx_i)-\psi(W_j(0)^Tx_i))$ as the $j$th entry of the bias vector. By Lemma \ref{lem:design-rf}, we know that the design matrix $\psi(X^TW(0)^T)$ satisfies Condition A and Condition B. So it suffices to bound $\frac{1}{p}\sum_{j=1}^p|b_j|$. With the help of Theorem \ref{thm:nn-grad}, we have
\begin{eqnarray*}
\frac{1}{p}\sum_{j=1}^p|b_j| &\leq& \frac{\gamma}{p^{3/2}}\sum_{j=1}^p\sum_{t=0}^{t_{\max}-1}\sum_{i=1}^n|y_i-u_i(t)||(W_j(t)-W_j(0))^Tx_i| \\
&\leq& \frac{R_1\gamma}{p^{1/2}}\sum_{t=0}^{t_{\max}-1}\sum_{i=1}^n|y_i-u_i(t)|\|x_i\| \\
&\leq& \frac{R_1\gamma}{p^{1/2}}\sum_{t=0}^{t_{\max}-1}\|y-u(t)\|\sqrt{\sum_{i=1}^n\|x_i\|^2} \\
&\lesssim& \frac{R_1}{p^{1/2}}\|y-u(0)\|\sqrt{\sum_{i=1}^n\|x_i\|^2} \\
&\lesssim& \frac{n^2\log p}{p},
\end{eqnarray*}
where the last inequality is by $\sum_{i=1}^n\|x_i\|^2\lesssim nd$ due to a standard chi-squared bound (Lemma \ref{lem:chi-squared}), and $\|u(0)\|^2\lesssim n$ due to Markov's inequality and $\mathbb{E}|u_i(0)|^2 = \mathbb{E}\Var(u_i(0)|X) \leq 1$. By Theorem \ref{thm:robust-reg-b} and Lemma \ref{lem:lim-G}, we have $\frac{1}{p}\|\wt{\beta}-\wh{\beta}\|^2 \lesssim \frac{n^3\log p}{p}$, which is the desired conclusion.
\end{proof}

\begin{proof}[Proof of Theorem \ref{thm:repair-nn-2}]
The analysis of $\wh{v}_1,...,\wh{v}_p$ is the same as that in the proof of Theorem \ref{thm:repair-nn-1}, and we have $\wt{W}_j=\wh{W}_j$ for all $j\in[p]$ with high probability.

To analyze $\wh{u}$, we apply Theorem \ref{thm:robust-reg}. It suffices to check Condition A' and Condition B for the design matrix $\psi(X^T\wt{W}^T)=\psi(X^T\wh{W}^T)$. To check Condition A', we consider i.i.d. Rademacher random variables $\delta_1,...,\delta_m$. Then, we define a different gradient update with initialization $\check{W}_j(0)=\delta_jW_j(0)$ and $\check{\beta}_j(0)=\delta_j\beta_j(0)$, and
\begin{eqnarray*}
\check{\beta}_j(t) &=& \check{\beta}_j(t-1) - \gamma\frac{\partial L(\beta,W)}{\partial \beta_j}|_{(\beta,W)=(\check{\beta}(t-1),\check{W}(t-1))}, \\
\check{W}_j(t) &=& \check{W}_j(t-1) - \frac{\gamma}{d}\frac{\partial L(\beta,W)}{\partial W_j}|_{(\beta,W)=(\check{\beta}(t),\check{W}(t-1))}.
\end{eqnarray*}
In other words, $(W(t),\beta(t))$ and $(\check{W}(t),\check{\beta}(t))$ only differ in terms of the initialization. Recall that $u_i(t)=\frac{1}{\sqrt{p}}\sum_{j=1}^p\beta_j(t)\psi(W_j(t)^Tx_i)$. We also define 
\begin{eqnarray*}
\check{u}_i(t) &=& \frac{1}{\sqrt{p}}\sum_{j=1}^p\check{\beta}_j(t)\psi(\check{W}_j(t)^Tx_i), \\
v_i(t) &=& \frac{1}{\sqrt{p}}\sum_{j=1}^p\beta_j(t)\psi(W_j(t-1)^Tx_i), \\
\check{v}_i(t) &=& \frac{1}{\sqrt{p}}\sum_{j=1}^p\check{\beta}_j(t)\psi(\check{W}_j(t-1)^Tx_i).
\end{eqnarray*}
It is easy to see that
$$\check{u}_i(t)=\frac{1}{\sqrt{p}}\sum_{j=1}^p\delta_j\beta_j(t)\psi(\delta_jW_j(t)^Tx_i)=\frac{1}{\sqrt{p}}\sum_{j=1}^p\beta_j(t)\psi(W_j(t)^Tx_i)=u_i(t).$$
Similarly, we also have
$$\check{v}_i(t)=\frac{1}{\sqrt{p}}\sum_{j=1}^p\delta_j\beta_j(t)\psi(\delta_jW_j(t-1)^Tx_i)=\frac{1}{\sqrt{p}}\sum_{j=1}^p\beta_j(t)\psi(W_j(t-1)^Tx_i)=v_i(t).$$
Suppose $\check{W}_j(k)=\delta_jW_j(k)$ and $\check{\beta}_j(k)=\delta_j\beta_j(k)$ are true. Since
\begin{eqnarray*}
\frac{\partial L(\beta,W)}{\partial \beta_j}|_{(\beta,W)=(\check{\beta}(k),\check{W}(k))} &=& \frac{1}{\sqrt{p}}\sum_{i=1}^n(\check{u}_i(k)-y_i)\psi(\check{W}_j(k)^Tx_i) \\
&=& \frac{1}{\sqrt{p}}\sum_{i=1}^n({u}_i(k)-y_i)\psi(\delta_j{W}_j(k)^Tx_i) \\
&=& \delta_j\frac{1}{\sqrt{p}}\sum_{i=1}^n({u}_i(k)-y_i)\psi({W}_j(k)^Tx_i) \\
&=& \delta_j\frac{\partial L(\beta,W)}{\partial \beta_j}|_{(\beta,W)=({\beta}(k),{W}(k))},
\end{eqnarray*}
we have $\check{\beta}_j(k+1)=\delta_j\beta_j(k+1)$.
Then,
\begin{eqnarray*}
\frac{\partial L(\beta,W)}{\partial W_j}|_{(\beta,W)=(\check{\beta}(k+1),\check{W}(k))} &=& \frac{1}{\sqrt{p}}\check{\beta}_j(k+1)\sum_{i=1}^n(\check{v}_i(k+1)-y_i)\psi'(\check{W}_j(k)^Tx_i)x_i \\
&=& \frac{1}{\sqrt{p}}\delta_j{\beta}_j(k+1)\sum_{i=1}^n({v}_i(k+1)-y_i)\psi'(\delta_j{W}_j(k)^Tx_i)x_i \\
&=& \frac{1}{\sqrt{p}}\delta_j{\beta}_j(k+1)\sum_{i=1}^n({v}_i(k+1)-y_i)\psi'({W}_j(k)^Tx_i)x_i \\
&=& \delta_j\frac{\partial L(\beta,W)}{\partial W_j}|_{(\beta,W)=({\beta}(k+1),{W}(k))},
\end{eqnarray*}
and thus we also have $\check{W}_j(k+1)=\delta_jW_j(k+1)$. A mathematical induction argument leads to $\check{W}_j(t)=\delta_jW_j(t)$ and $\check{\beta}_j(t)=\delta_j\beta_j(t)$ for all $t\geq 1$. Since $(\check{W}(0),\check{\beta}(0))$ and $(W(0),\beta(0))$ have the same distribution, we can conclude that $(\check{W}(t),\check{\beta}(t))$ and $(W(t),\beta(t))$ also have the same distribution. Therefore, Condition A' holds for the design matrix $\psi(X^T\wh{W}^T)=\psi(X^TW(t_{\max})^T)$.

We also need to check Condition B. By Theorem \ref{thm:nn-grad}, we have
\begin{eqnarray*}
&& \left|\frac{1}{p}\sum_{j=1}^p\left|\sum_{i=1}^n\psi(\wh{W}_j^Tx_i)\Delta_i\right| - \frac{1}{p}\sum_{j=1}^p\left|\sum_{i=1}^n\psi(W_j(0)^Tx_i)\Delta_i\right|\right| \\
&\leq& \frac{1}{p}\sum_{j=1}^p\sum_{i=1}^n|\wh{W}_j^Tx_i-W_j(0)^Tx_i||\Delta_i| \\
&\leq& R_1\sum_{i=1}^n\|x_i\||\Delta_i| \leq R_1\sqrt{\sum_{i=1}^n\|x_i\|^2} \lesssim \frac{n^{3/2}\log p}{\sqrt{p}},
\end{eqnarray*}
where $\sum_{i=1}^n\|x_i\|^2\lesssim nd$ is by Lemma \ref{lem:chi-squared}. By Lemma \ref{lem:design-rf}, we can deduce that
$$\inf_{\|\Delta\|=1}\frac{1}{p}\sum_{j=1}^p\left|\sum_{i=1}^n\psi(\wh{W}_j^Tx_i)\Delta_i\right|\gtrsim 1,$$
as long as $\frac{n^{3/2}\log p}{\sqrt{p}}$ is sufficiently small. 
According to Lemma \ref{lem:lim-G}, we also have
$$\sup_{\|\Delta\|=1}\frac{1}{p}\sum_{j=1}^p\left|\sum_{i=1}^n\psi(\wh{W}_j^Tx_i)\Delta_i\right|^2\lesssim 1+\frac{n^2\log p}{\sqrt{p}}.$$
See (\ref{eq:Gk-spec-arctan}) for details of derivation.
Therefore, Condition B holds with $\overline{\lambda}^2\asymp 1+\frac{n^2\log p}{\sqrt{p}}$ and $\underline{\lambda}\asymp 1$. Apply Theorem \ref{thm:robust-reg}, we have $\wt{\beta}=\wh{\beta}$ with high probability as desired.
\end{proof}
